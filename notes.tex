\newcount\Comments  
\Comments=1   
\documentclass[a4paper,12pt, english]{article}
\usepackage[top=2cm, bottom=2cm, left=2cm, right=2cm]{geometry}

\usepackage{babel}
%\usepackage{amsmath}

\usepackage{color}

\definecolor{darkgreen}{rgb}{0,0.5,0}
\definecolor{purple}{rgb}{1,0,1}
%\usepackage{subcaption}

\newcommand{\kibitz}[2]{\ifnum\Comments=1\textcolor{#1}{#2}\fi}
% add yourself here:
\newcommand{\ls}[1]{\kibitz{red}      {[Larisa: #1]}}
\newcommand{\cg}[1]  {\kibitz{purple}   {[Crina: #1]}}
\newcommand{\ns}[1]{\kibitz{cyan}     {[Noureddin: #1]}}



\usepackage{listings}
\usepackage{url}
%\usepackage{graphicx}

\usepackage{verbatim}

%\usepackage{caption}
%\usepackage{enumitem}

%\onehalfspacing

\begin{document}

\title{A List of Suggestions/Notes/Ideas}
%\date{Mar 2014}
%\author{By: Noureddin Sadawi}
%\maketitle

\large
\section{By Noureddin}
\begin{enumerate}
	\item Please create your own section (copy and paste this \LaTeX  source :-) )
	\item Add your thoughts/suggestions/comments
	\item Add dates so we keep up to date with progress
	\item Look at this example: \ns{This is how you comment on something!}
	\item For Larisa it's \ls{I love Japanese tea with rice!}
	\item For Crina it's  \cg{I love Japanese rice with tea!}
\end{enumerate}    

\section{By Larisa}
\begin{enumerate}
	\item We have analysed available descriptors from various sources (bottom-up approach), and there are too many of them. We now will try top-down approach. We will consider several datasets, from Dundee and publicly available ones, e.g. from ChEMBL and also several QSAR studies. We will try to annotate them with some descriptors that would convey what is important to record about those datasets and QSAR- specific studies. We then will come up with a set of descriptors that are suited for our task. We will check if what existing resources have the required descriptors, and if necessary we will define new ones (see the next point).
	\item Crina suggested that we can develop new descriptors/ measures that fit better to support our task.
	\item We will target to come up with ~50 descriptors and we will evaluate them with experts through questionary and select  ~ 20 the most popular descriptors.
	\item We will also put our questionary to a public domain and give an opportunity to everyone to comment on descriptors. Such an approach would ensure that we identify a set of useful descriptors.
	\end{enumerate}  
		
\end{document}
